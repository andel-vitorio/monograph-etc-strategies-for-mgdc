\chapter{Considerações Finais} \label{cap5}

Este trabalho apresentou um estudo sobre o controle baseado em eventos aplicado à conversores \acrshort{cc}-\acrshort{cc} \textit{Buck} e \textit{Boost} presentes em \acrshortpl{mr} de \acrshort{cc}. O estudo também analisou as influências de \acrshortpl{crl} e \acrshortpl{cpl} quando conectadas a esses conversores. Com o objetivo de assegurar a estabilidade no ponto de operação de conversores em malha fechada controlados por realimentação dos estados com menor quantidade de eventos gerados, foi proposta uma condição suficiente, baseada na teoria da estabilidade de \textit{Lyapunov}, para o projeto de \acrshortpl{etm} estáticos e dinâmicos por meio da abordagem por \textit{co-design} visando minimizar o número de eventos gerados.

Para avaliar o desempenho e o comportamento dos conversores sob diferentes \acrshortpl{etm} projetados utilizando a condição proposta, foram realizadas diversas simulações em diferentes cenários e situações. Essas simulações utilizaram os modelos matemáticos desenvolvidos neste trabalho, os quais descrevem o comportamento dinâmico dos conversores \textit{Buck} e \textit{Boost} conectados tanto a uma \acrshort{cpl} quanto a uma \acrshort{crl}, além dos modelos dos conversores em malha fechada com o \acrshort{etc}. Para uma análise quantitativa, foram obtidos as médias dos \acrshortpl{itee}, os tempos de acomodação dos sinais de saída e os indices de desempenho \acrshort{ise}, \acrshort{itse} e \acrshort{isc}.

Observou-se que os \acrshortpl{etm} dinâmicos apresentaram uma menor quantidade de eventos acionados em comparação com os \acrshortpl{etm} estáticos. Desta forma, os \acrshortpl{etm} dinâmicos apresentaram maiores médias de \acrshortpl{itee} em relação aos estáticos. Além disso, constatou-se que um conversor operando sob um \acrshort{etm} estático e, nas mesmas condições, sob um \acrshort{etm} dinâmico, apresentou o mesmo comportamento nos sinais de saída, conforme evidenciado pelos índices de desempenho que foram semelhantes ao comparar o \acrshort{etm} estático e dinâmico. Portanto, o \acrshort{etm} se mostra como uma alternativa mais interessante que o estático, pois consegue obter o mesmo comportamento, porém com uma menor quantidade de eventos gerados.

Também é notável que, ao operar em torno de pontos instáveis, os conversores sob os \acrshortpl{etm} projetados garantiram a estabilidade dos pontos de operação em malha fechada, convergindo os sinais de saída em direção aos pontos de operação definidos. No caso dos pontos estáveis, independentemente dos \acrshortpl{etm} projetados, houve uma geração muito reduzida de eventos, resultando em um sinal de entrada constante por períodos prolongados. Quando a potência da \acrshort{cpl} sofria variações, notou-se que os conversores permaneciam estáveis; contudo, estabilizavam-se em pontos diferentes dos pontos de operação definidos. Quanto maior a variação, mais distante os conversores estabilizavam do ponto de operação definido.

Foi observado que é importante selecionar valores ótimos para os parâmetros no projeto da lei de acionamento do \acrshort{etm} dinâmico, a fim de alcançar uma resposta do sistema mais eficaz. Observou-se que, à medida que esses parâmetros aumentavam, o \acrshort{etm} dinâmico se aproximava mais do estático, resultando em uma maior frequência de acionamento de eventos. Consequentemente, o aumento na ocorrência de eventos resultava em um prolongamento do tempo necessário para os sinais de saída atingirem a estabilidade, acarretando, assim, em um aumento nos tempos de acomodação.

Ao realizar uma comparação entre os comportamentos dos conversores sob o controle do \acrshort{etc} e aqueles utilizando o critério de estabilidade de \textit{Hurwitz}, observou-se uma diferença significativa. Nem sempre o controlador baseado no critério de estabilidade de \textit{Hurwitz} foi capaz de estabilizar os modelos não lineares, resultando em oscilações frequentes nos sinais de saída e tempos de acomodação mais longos. Portanto, os \acrshortpl{etm} projetados demonstraram um desempenho superior.

Desta forma, diante dos desafios das \acrshortpl{mr}, o \acrshort{etc} pode ser uma alternativa promissora ao permitir uma resposta aperiódica, gerando eventos somente quando necessário, o que otimiza o uso de recursos e diminui a carga computacional. Essa otimização também contribui para aprimorar a confiabilidade, a eficiência e a resiliência das \acrshortpl{mr}.