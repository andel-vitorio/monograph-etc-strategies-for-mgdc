\chapter{Controle Baseado em Eventos de Conversores CC-CC} \label{cap4}

% Essa análise fornece insights sobre a aplicação das abordagens de ETC e permite comparar suas vantagens e desvantagens.


\section{Modelo de \acrshort{etm} para Sistemas Lineares}

% ETM: Apresentação do sistema dinâmico
Nesta pesquisa, é proposto um modelo de \acrshort{etm} dinâmico para sistemas lineares, representada pela equação diferencial abaixo: \begin{equation} \dot{x}(t) = Ax(t) + Bu(t). \label{eq:linear_system_etm}\end{equation} Nesta equação, $ x(t) \in \mathbb{R}^n $ representa o estado do sistema, $ u(t) \in \mathbb{R}^m $ é a entrada de controle, e $ A \in \mathbb{R}^{n \times n} $ e $ B \in \mathbb{R}^{n \times m}$ são a matriz de estados e a matriz de entrada, respectivamente. Além disto, a origem do sistema é considerada o ponto de equilíbrio de interesse.

% ETM: Apresentação do erro de transmissão
Como discutido na seção \ref{section:etm_classification}, quando uma amostra de dados é transmitida no tempo de evento $t_k$, o estado disponível para o controlador é atualizado para $\hat{x}(t) = x(t_k)$, para todo $t$ no intervalo $[t_k, t_{k+1})$. Ao utilizar um \acrshort{zoh}, $\hat{x}(t)$ é mantido constante até o próximo tempo de evento $t_{k+1}$, resultando no erro de transmissão representado por: \begin{equation}
  e(t) = \hat{x}(t) - x(t), \quad \forall t \in [t_k, t_{k+1}).
\end{equation} Este erro ocorre durante o intervalo de tempo entre eventos, $ [t_k, t_{k+1})$.

% Adicionar um conclusão

\subsection{\acrshort{etm} Estático e Dinâmico}

% Resumo sobre o ETM estático
Como discutido anteriormente, o \acrshort{etm} estático opera considerando apenas os estados atuais do sistema $x(t)$ e o erro de transmissão $e(t)$. A sua lei de acionamento de eventos é definida como: \begin{equation} t_0 = 0, t_{k+1} = \inf \{t > t_k : \Gamma(x(t), e(t)) < 0 \}, \, \forall k \in \mathbb{N},\end{equation} onde $\Gamma(x, e)$ representa a função de evento do \acrshort{etm}. Adicionalmente, para uma classe específica de funções $\Gamma$ dada por \begin{equation}
  \Gamma(x(t), e(t)) = \sigma \alpha(\|x(t)\|) - \beta(\|e(t)\|),
\end{equation} com $\sigma \in (0,1)$, o sistema em malha fechada é assintoticamente estável.

% Função evento proposta
Com base nisso, é proposta uma função de evento onde $\alpha(\|x(t)\|) = x^T(t)\Psi x(t)$, $\beta(\|e(t)\|) = e^T(t)\Xi e(t)$ e $\sigma = 1$, ou seja: \begin{equation}
  \Gamma(x(t), e(t)) = x^T(t)\Psi x(t) - e^T(t)\Xi e(t),
  \label{eq:static_etm_gamma}
\end{equation} onde $\Psi, \, \Xi \in \mathbb{R}^{n \times n}$ e a condição suficiente para o projeto do \acrshort{etm} por co-design para o \acrshort{etc} contínuo e o controlador por realimentação de estados é declarada a seguir:

\begin{theorem}
  \label{theorem:constraints-2}
  Se existirem matrizes semidefinidas positivas $\Xi, \Psi, X \in \mathbb{R}$ e uma matriz $\tilde{K} \in \mathbb{R}^{m \times n}$ que satisfaçam a seguinte \acrshort{lmi}:
  \begin{equation}
    \begin{bmatrix}
      \mathrm{He}(AX +B\tilde{K}) & B\tilde{K}   & X             \\
      \star              & -\tilde{\Xi} & 0             \\
      \star              & \star        & -\tilde{\Psi}
    \end{bmatrix} < 0,
    \label{eq:etm_lmi_1}
  \end{equation}
  então, a origem do sistema de malha fechada é assintoticamente estável com $K = \tilde{K}X^{-1}$, $\Xi= X^{-1}\tilde{\Xi}X$, $\Psi = \tilde{\Psi}^{-1}$, $P = X^{-1}$ e a função de Lyapunov $V(x)=x^TPx$.
\end{theorem}

% ETM: Tempo de ocorrência dos eventos e a variável interna dinâmica
No \acrshort{etm} dinâmico, a lei de acionamento de eventos é definida como: \begin{equation} t_0 = 0, t_{k+1} = \inf \{t > t_k : \eta(t) + \theta \Gamma(x(t), e(t) < 0) \}, \, \forall k \in \mathbb{N} \end{equation} onde $\theta \in \mathbb{R}_{\geq 0}$ é um parâmetro de projeto, a função de evento $\Gamma(x, e)$ é a mesma definida para o \acrshort{etm} estático, em \eqref{eq:static_etm_gamma} e $\eta$ é a variável dinâmica definida por: \begin{equation}  \dot{\eta} = - \lambda \eta(t) + \Gamma(x(t), e(t)), \label{eq:dynamic-etm}\end{equation} onde $\lambda \in R_{\geq} 0 $ é o parâmetro de projeto relacionado à taxa de decaimento de $\eta(t)$.

% ETM: Condições de Co-design
\subsection{Otimização do \acrshort{etm}}


Para reduzir o número de eventos gerados, é proposto o seguinte problema de otimização convexa que visa minimizar $\lambda_{\max} (\Xi)$ e maximizar $\lambda_{\min}(\Psi)$: \begin{equation}\underset{Q, X, \tilde{\Xi}, \tilde{\Psi}, \tilde{K}}\min \quad \mathbf{tr}(\tilde{\Xi} + \tilde{\Psi} + Q). \label{eq:optimization_problem}\end{equation} Este problema é sujeito a \acrshort{lmi} apresentada em \eqref{eq:etm_lmi_1} e à seguinte \acrshort{lmi}: \begin{equation}\begin{bmatrix}
  -Q & I \\ \star & -X
\end{bmatrix} < 0. \label{eq:constraints_2}\end{equation} A solução deste problema tende a minimizar os autovalores de $Q$, $\tilde{\Xi}$ e $\tilde{\Psi}$ e a aumentar o intervalo de tempo entre os eventos. Se o problema for factível, é possível determinar as variáveis do problema e obter os parâmetros de projeto do \acrshort{etm} capazes de reduzir o número de transmissões geradas pelo \acrshort{etm}.