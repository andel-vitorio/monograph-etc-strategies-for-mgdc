\lstset{
	language=Python,
	tabsize=3,
	basicstyle=\small\linespread{0.8}\ttfamily,
	keywordstyle=\bfseries\color{blue},
	commentstyle=\color{green!40!black},
	frame=single,
	breaklines=false,
	breakatwhitespace=false
}

\pagenumbering{arabic}

\chapter{Introdução}

\section{Microrredes CC e Estratégias de Controle em Rede}

Microrredes isoladas são sistemas elétricos de pequeno porte, com geração distribuída, armazenamento e carga local. Elas apresentam características notavelmente distintas do sistema de energia convencional, especialmente no que tange ao controle de rede. A peculiaridade da natureza CC das microrredes traz consigo desafios específicos, demandando a formulação e implementação de estratégias de controle eficazes. A seguir, aprofundaremos alguns aspectos cruciais que ilustram a complexidade intrínseca desse cenário dinâmico.

Um dos principais desafios do controle de microrredes isoladas é a falta de sincronismo. Em sistemas convencionais, a frequência da rede é mantida constante por um gerador sincronizado. No entanto, em microrredes isoladas, a geração distribuída é geralmente composta por fontes renováveis, como painéis solares e turbinas eólicas, que não são sincronizadas. Isso pode levar a flutuações na frequência da rede, que podem prejudicar o desempenho dos equipamentos e até mesmo causar falhas. Outro desafio é a falta de estabilidade. Microrredes isoladas são sistemas mais sensíveis a perturbações do que sistemas convencionais. Isso ocorre porque elas têm menor capacidade de absorção de energia de reserva, o que as torna mais suscetíveis a oscilações.

Apesar dos desafios, microrredes isoladas também oferecem oportunidades para o controle de rede. Por exemplo, a geração distribuída pode ser usada para melhorar a qualidade da energia. As fontes renováveis são geralmente mais consistentes do que as fontes convencionais, o que pode ajudar a reduzir a flutuação da tensão e da frequência da rede. Além disso, microrredes isoladas podem ser usadas para aumentar a confiabilidade do fornecimento de energia. A geração distribuída pode fornecer energia local, mesmo em caso de falha na rede principal.

A inserção de recursos energéticos distribuídos (REDs) nas microrredes, como geração solar e eólica, traz desafios operacionais, como a garantia da estabilidade do sistema e o compartilhamento eficiente de potência. Para enfrentar esses desafios, recentes estudos propuseram esquemas de controle hierárquico para microrredes, com foco primordial na provisão de reserva. A reserva é a capacidade de um sistema elétrico de fornecer potência adicional para compensar eventos inesperados, como falhas de geração ou cargas inesperadas. Ela desempenha um papel crucial na robustez das microrredes isoladas, que não possuem conexão com a rede elétrica principal.

A provisão de reserva emerge como fator crucial em microrredes isoladas, garantindo a continuidade do fornecimento de energia diante de falhas ou variações na demanda. A reserva pré-primária, ao tirar proveito da inércia rotacional dos geradores, mantém a estabilidade da rede, enquanto a reserva primária envolve fontes controláveis, ajustando-se conforme necessário para fornecer potência adicional.

Os esquemas de controle hierárquico para microrredes, compostos por três níveis - primário, secundário e terciário - desempenham funções específicas para aprimorar o desempenho global. O nível primário visa assegurar a estabilidade local, monitorando e ajustando a frequência e a tensão para mantê-las dentro de limites aceitáveis. No nível secundário, a coordenação das ações das unidades geradoras é priorizada para garantir o compartilhamento eficiente de potência, visando minimizar perdas e otimizar a eficiência do sistema. Por fim, o nível terciário assume a responsabilidade de coordenar as ações tanto das unidades geradoras quanto dos consumidores, almejando a minimização dos custos operacionais. Esses esquemas hierárquicos, ao atuarem de maneira integrada, demonstram o potencial de aprimorar significativamente a confiabilidade e eficiência das microrredes isoladas.

A crescente inserção de recursos energéticos distribuídos (REDs) nas microrredes CC traz desafios operacionais, como a necessidade de garantir a confiabilidade e eficiência do sistema. O controle centralizado tradicional é limitado em sua capacidade de resposta a eventos imprevisíveis, como falhas de componentes ou variações na demanda. Diante desse cenário, as Estratégias de Controle Acionado por Evento (ETCs) emergem como uma alternativa promissora para o controle de microrredes CC. As ETCs realizam tarefas de controle em resposta a eventos específicos, permitindo uma alocação de recursos mais eficiente e uma resposta mais rápida a mudanças nas condições operacionais.

As Estratégias de Controle Acionado por Evento (ETCs) apresentam diversas vantagens, como eficiência na alocação de recursos, rápida resposta a mudanças nas condições operacionais e alta adaptabilidade. Essas vantagens são possíveis porque as ETCs realizam tarefas de controle apenas quando necessário, o que reduz o desperdício de recursos e garante a confiabilidade do sistema. Além disso, as ETCs são altamente adaptáveis, podendo ser ajustadas a mudanças nas condições operacionais ou na configuração do sistema.

No entanto, as ETCs também enfrentam desafios, como o projeto de gatilhos eficazes e o desenvolvimento de sistemas adaptativos confiáveis. O projeto de gatilhos eficazes é essencial para garantir o desempenho das ETCs, pois eles são responsáveis por determinar quando as tarefas de controle devem ser executadas. O desenvolvimento de sistemas adaptativos confiáveis também é um desafio técnico, pois é necessário garantir que esses sistemas sejam capazes de se adaptar às mudanças nas condições operacionais sem comprometer a confiabilidade do sistema. Apesar desses desafios, as ETCs apresentam um potencial significativo para aprimorar o desempenho das microrredes CC, especialmente em aplicações críticas em tempo real. As ETCs podem ajudar a reduzir o consumo de energia, melhorar a resposta a eventos imprevisíveis e aumentar a robustez dos sistemas.

Com base na análise das microrredes CC e considerando as particularidades das estratégias de Controle Acionado por Evento (ETCs), Este trabalho apresenta contribuições significativas para o controle de microrredes CC. A principal inovação é a definição de propriedades específicas para a verificação da segurança e eficiência dessas microrredes. Além disso, a proposta contempla a criação de modelos que abrangem a regulação de tensão e aprimoramento da qualidade de energia, contribuindo para a estabilidade local, compartilhamento eficiente de potência e otimização econômica. Essas contribuições exploram novos horizontes no campo do controle de microrredes CC e sublinham a importância de estratégias formais para garantir a segurança, eficiência e adaptabilidade desses sistemas em constante evolução.

\section{Objetivos}

\subsection{Objetivos Gerais}

O objetivo deste trabalho é avaliar e aprimorar estratégias de controle acionado por eventos (ETC) em microrredes de corrente contínua (CC), visando a otimização do funcionamento desses sistemas em ambientes descentralizados e voltados para redes distribuídas. A pesquisa se propõe a realizar uma análise abrangente das estratégias de ETC, explorando tecnologias recentes e investigando sua aplicabilidade específica nas microrredes de CC. O foco principal é contribuir para o fornecimento de energia eficiente e confiável nesse contexto dinâmico.

\subsection{Objetivos Específicos}
Para alcançar o objetivo geral, serão delineados os seguintes objetivos específicos que direcionarão as atividades do projeto:

\begin{enumerate}
	\item[(1)] Desenvolver um modelo específico de microrrede CC
	\item[(2)] Implementar um simulador de uma microrrede a partir do modelo desenvolvido
	\item[(3)] Projetar e avaliar diferentes estratégias de controle baseados no ETC para a microrrede CC
	\item[(4)] Aplicar LMIs para análise e controle do sistema
\end{enumerate}

\section{Organização do Trabalho}

O presente trabalho encontra-se estruturado em quatro capítulos dos quais, o primeiro é composto por esta introdução ao trabalho.

O Capítulo \ref{cap2}, "Revisão Bibliográfica", abrange a fundamentação teórica, destacando o Sistema de Controle ETC, Microrredes CC e LMIs. A segunda parte revisita Trabalhos Relacionados, focando em Controle Acionado por Eventos e Controle em Rede, estabelecendo bases para o projeto.

O Capítulo \ref{cap3}, "Modelos e Protótipo", destaca a criação de modelos específicos para microrredes CC e aborda o desenvolvimento do protótipo, essencial para a implementação e teste das estratégias de Controle Acionado por Eventos propostas.

O Capítulo \ref{cap4}, "Avaliação de ETC", concentra-se na análise e avaliação das estratégias de Controle Acionado por Eventos (ETC) implementadas. Este capítulo examina o desempenho das estratégias em microrredes de corrente contínua, considerando métricas como eficiência energética, confiabilidade do sistema e adaptabilidade às condições operacionais.

O Capítulo \ref{cap5}, "Conclusão", apresenta as principais descobertas do projeto, destacando conclusões derivadas da avaliação das estratégias de Controle Acionado por Eventos em microrredes de corrente contínua.

% Fim Capítulo