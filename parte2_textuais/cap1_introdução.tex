\lstset{
  language=Python,
  tabsize=3,
  basicstyle=\small\linespread{0.8}\ttfamily,
  keywordstyle=\bfseries\color{blue},
  commentstyle=\color{green!40!black},
  frame=single,
  breaklines=false,
  breakatwhitespace=false
}

\pagenumbering{arabic}

\chapter{Introdução}

% To-do: Verificar o título
\section{Microrredes e Estratégias de Controle por Eventos}

% Tópico: Definição das Microrredes
Microrredes são sistemas elétricos de pequeno porte, com geração distribuída, armazenamento e carga local. Geralmente, possuem fontes renováveis, como painéis solares e turbinas eólicas, que fornecem energia para cargas locais; e podem funcionar de forma isolada ou conectada à rede elétrica principal. Elas apresentam características notavelmente distintas dos sistemas convencionais de energia, especialmente no que tange ao controle das redes de distribuição. \cite{Paigi2013}. A peculiaridade da natureza de uma microrrede traz consigo desafios específicos, demandando a formulação e implementação de estratégias de controle eficazes. A seguir, aprofundaremos alguns aspectos cruciais que ilustram a complexidade intrínseca desse cenário dinâmico.

% Tópico: Estrutura da Microrrede
Com base nas particularidades das microrredes e nos desafios de controle que elas apresentam, é crucial entender o esquema hierárquico de controle que é implementado. Esse esquema é composto por três níveis - primário, secundário e terciário - cada um desempenhando funções específicas para aprimorar o desempenho global. No nível primário, o foco está em garantir a estabilidade local, monitorando e ajustando a frequência e a tensão para mantê-las dentro de limites aceitáveis. \cite{Paigi2013}. Em seguida, no nível secundário, a coordenação das ações das unidades geradoras é priorizada para garantir o compartilhamento eficiente de potência, visando minimizar perdas e otimizar a eficiência do sistema. \cite{Paigi2013}. Por fim, o nível terciário assume a responsabilidade de coordenar as ações tanto das unidades geradoras quanto dos consumidores, buscando a minimização dos custos operacionais. \cite{Paigi2013}. A integração desses esquemas hierárquicos demonstra o potencial de significativamente aprimorar a confiabilidade e eficiência das microrredes.

% Tópico: Apresentação dos Desafios das Microrredes
Representando um avanço na descentralização da geração de energia, o controle das microrredes enfrenta desafios específicos. Primeiramente, a ausência de inércia rotativa, essencial nos sistemas elétricos convencionais, é um desafio central. Nas microrredes, essa falta de inércia é devida à predominância de elementos eletronicamente interconectados. \cite{Paigi2013}. Em segundo lugar, a dependência da amplitude de tensão nas linhas de energia das redes de baixa ou média tensão adiciona complexidade ao controle da potência ativa, demandando estratégias adaptativas. \cite{Paigi2013}. Por fim, a intermitência das fontes renováveis, como solar e eólica, apresenta um terceiro desafio, exigindo controle dinâmico para garantir a estabilidade do sistema. \cite{Paigi2013}. Superar esses desafios é essencial para a efetiva operação das microrredes, destacando a necessidade de soluções inovadoras no controle de energia elétrica.

% Tópico: Apresentação dos Desafios das Microrredes: REDs
A integração de recursos energéticos distribuídos (\acrshortpl{red}), como painéis solares e turbinas eólicas, amplia os desafios para as microrredes isoladas, que não possuem conexão com a rede elétrica principal, demandando mecanismos eficientes para o compartilhamento de potência e a manutenção da estabilidade da rede. \cite{Paigi2013}. Nesse contexto, esquemas de controle hierárquico com foco na provisão de reserva surgem como soluções promissoras. A reserva, que é a capacidade de fornecer potência adicional para compensar eventos inesperados, é crucial para a robustez das microrrede isoladas, e pode ser dividida em reserva pré-primária e reserva primária por meio dos elementos da microrrede. \cite{Paigi2013}. A reserva pré-primária, aproveitando a inércia rotacional dos geradores, contribui para a manutenção da estabilidade da rede, enquanto a reserva primária envolve fontes controláveis que se ajustam conforme necessário para fornecer potência adicional. \cite{Paigi2013}. Garantir a provisão adequada de reserva assegura a continuidade do fornecimento de energia, mesmo diante de falhas ou variações na demanda, o que promove a confiabilidade e a resiliência das microrredes isoladas.

% Tópico: Redes Elétricas Inteligentes e Integração de Recursos de Comunicação
% To do: verificar o termo (microrredes em rede)
As redes elétricas inteligentes (\textit{smart grids}) são redes que cada vez mais integram recursos energéticos distribuídos. Isso aumenta a eficiência e a sustentabilidade da energia, mas também traz desafios de controle e operação, como a restauração de frequência e tensão, o compartilhamento de potência ativa e reativa, o balanceamento do estado de carga das baterias, a operação econômica ideal e a comutação suave. \cite{Zhou2020}. As microrredes surgem como soluções promissoras para superar esses desafios, podendo ser interconectadas para formar microrredes em rede (\acrshort{nmg}, do inglês \textit{Networked Microgrids}). Neste contexto, a confiabilidade das redes de comunicação entre as unidades das microrredes desempenha um papel crucial no sucesso das \acrshort{nmg}, pois problemas na comunicação podem comprometer o desempenho do controle do sistema, resultando em uma redução da eficiência e instabilidade operacional. \cite{Zhou2020}.

% Tópico: Controle Acionado por Evento - Introdução
Em cenários com recursos de comunicação e energia são essenciais, e demandam uma gestão cuidadosa para evitar problemas que possam comprometer o sistema, o controle acionado por eventos (\acrshort{etc}, do inglês \textit{event-triggered control}) surge como uma solução inovadora para o gerenciamento eficiente das \acrshortpl{mr}. Diferentemente das abordagens tradicionais que utilizam intervalos fixos de amostragem e comunicação, o \acrshort{etc} adapta esses processos em resposta a eventos específicos, como mudanças na demanda de energia ou no estado da rede. \cite{coutinho2021}. Esta adaptabilidade permite uma utilização mais eficiente dos recursos disponíveis ao reduzir os desperdícios, realizando amostragem e comunicação apenas quando necessário. Desta forma, o \acrshort{etc} pode auxiliar na redução do consumo de energia, da comunicação requerida e do processamento computacional dos sistemas. \cite{coutinho2021}.


% Tópico: Controle Acionado por Evento - Classificações - Dinâmico e Estático
% To-do: Verificar o termo: função de ativação (trigger-function)
As estratégias de \acrshort{etc} podem ser classificadas com base na função de ativação presente no mecanismo de ativação de eventos (\acrshort{etm}, do inglês \textit{Event-Triggering Mechanism}), o qual é responsável por desencadear os eventos de controle do sistema, que pode ser estática ou dinâmica (ou adaptativa). No caso do \acrshort{etc} estático, os eventos de controle são definidos usando a função de ativação estática, que depende somente da medição do estado atual do sistema e do último estado transmitido. \cite{coutinho2021}. O \acrshort{etc} dinâmico ou adaptativo incorpora uma variável dinâmica interna adicional, onde os mecanismos de acionamento dos eventos são constituídos de relógios internos ou variáveis dinâmicas que funcionam como relógios, cuja taxa de crescimento está relacionada ao estado do sistema. \cite{Girard2015}.

% Tópico: Controle Acionado por Evento - Classificações - Emulação e Co-design
Além disso, existem duas abordagens principais para o projeto de sistemas \acrshort{etc}: emulação e co-design. Na abordagem por emulação, o projeto é dividido em duas etapas. Primeiramente, um controlador é projetado para garantir a estabilidade ou o desempenho desejado do sistema em malha fechada sem considerar o \acrshort{etm} ou a rede de comunicação. Em seguida, o \acrshort{etm} é projetado considerando o controlador já existente e os efeitos induzidos pela presença de uma rede de comunicação. Isso oferece flexibilidade no projeto do \acrshort{etm}, mas pode limitar o desempenho em malha fechada e exigir mais transmissões do que o necessário. \cite{coutinho2021}.

Por outro lado, o co-design envolve o projeto simultâneo do \acrshort{etm} e da lei de controle para superar as limitações da abordagem de emulação. No entanto, o co-design é desafiador devido a problemas de otimização não convexas ou multiobjetivo, e a análise geralmente é limitada a classes específicas de controladores e ETMs. \cite{coutinho2021}.

% Tópico: Conclusão
% To-do: Analisar novamente a conclusão
Com base na análise das \acrshortpl{mr} e considerando as diferentes estratégias \acrshort{etc}, este estudo oferece contribuições para o controle das \acrshortpl{mr}. Além disso, a proposta inclui a elaboração de modelos abrangendo a regulação de tensão e a melhoria da qualidade de energia, visando promover a estabilidade local, compartilhamento eficiente de potência e otimização econômica. Essas contribuições exploram novas perspectivas no âmbito do controle de \acrshortpl{mr} e ressaltam a importância de estratégias formais para assegurar a segurança, eficiência e adaptabilidade desses sistemas em constante evolução.


\section{Objetivos}

\subsection{Objetivos Gerais}

% To-do: Analisar novamente os objetivos gerais
Neste estudo, propomos avaliar estratégias \acrshort{etc} em microrredes, com o objetivo de otimizar o funcionamento desses sistemas em ambientes voltados para redes distribuídas. Para abstrair uma microrrede, será utilizado um conversor de corrente contínua (\acrshort{cc}) para corrente contínua. A pesquisa visa realizar uma análise abrangente das estratégias de ETC, explorando tecnologias recentes e investigando sua aplicabilidade específica nas microrredes. O foco principal é contribuir para o fornecimento de energia eficiente e confiável nesse contexto dinâmico.

\subsection{Objetivos Específicos}
Para alcançar o objetivo geral, serão delineados os seguintes objetivos específicos que direcionarão as atividades do projeto:

% To-do: verificar com os objetivos específicos novamente
\begin{enumerate}
  \item[(1)] Desenvolver um modelo específico de um conversor CC-CC
  \item[(2)] Implementar um simulador do conversor a partir do modelo desenvolvido
  \item[(3)] Projetar ETM e controlador para o conversor
  \item[(4)] Projetar e avaliar diferentes estratégias de controle baseados no ETC para a microrrede CC
\end{enumerate}

\section{Organização do Trabalho}

O presente trabalho encontra-se estruturado em quatro capítulos dos quais, o primeiro é composto por esta introdução ao trabalho.

O Capítulo \ref{cap2}, "Revisão Bibliográfica", abrange a fundamentação teórica, destacando o Sistema de Controle ETC, Microrredes CC e LMIs. A segunda parte revisita Trabalhos Relacionados, focando em Controle Acionado por Eventos e Controle em Rede, estabelecendo bases para o projeto.

O Capítulo \ref{cap3}, "Modelos e Protótipo", destaca a criação de modelos específicos para microrredes CC e aborda o desenvolvimento do protótipo, essencial para a implementação e teste das estratégias de Controle Acionado por Eventos propostas.

O Capítulo \ref{cap4}, "Avaliação de ETC", concentra-se na análise e avaliação das estratégias de Controle Acionado por Eventos (ETC) implementadas. Este capítulo examina o desempenho das estratégias em microrredes de corrente contínua, considerando métricas como eficiência energética, confiabilidade do sistema e adaptabilidade às condições operacionais.

O Capítulo \ref{cap5}, "Conclusão", apresenta as principais descobertas do projeto, destacando conclusões derivadas da avaliação das estratégias de Controle Acionado por Eventos em microrredes de corrente contínua.

% Fim Capítulo