\pagenumbering{arabic}

\chapter{Introdução}

\section{Contextualização}

% Tópico: Microrredes - Introdução
Microrredes se configuram como sistemas elétricos de pequeno porte, caracterizados por geração distribuída, armazenamento e consumo local de energia. Frequentemente, integram fontes renováveis, como painéis solares e turbinas eólicas, para suprir a demanda energética local, podendo operar de maneira isolada ou conectada à rede elétrica principal. As microrredes apresentam características singularmente distintas dos sistemas de energia tradicionais, principalmente no que concerne ao controle das redes de distribuição \citep{Paigi2013}. Essa natureza peculiar impõe desafios específicos, exigindo o desenvolvimento e implementação de estratégias de controle robustas e eficazes. A seguir, exploraremos em detalhes alguns aspectos cruciais que ilustram a complexa dinâmica deste cenário dinâmico.

% Tópico: Estrutura da Microrrede
As particularidades das microrredes e os desafios de controle inerentes a elas exigem a implementação de um esquema hierárquico de controle. Essa estrutura, composta por três níveis distintos - primário, secundário e terciário - visa aprimorar o desempenho global da microrrede. No nível primário, o foco reside na estabilidade local. Através do monitoramento e ajuste da frequência e da tensão, os controladores primários garantem que esses parâmetros operem dentro de limites aceitáveis \cite{Paigi2013}. O nível secundário prioriza a coordenação das unidades geradoras. Através de mecanismos de controle, o compartilhamento eficiente de potência é estabelecido, minimizando perdas e otimizando a eficiência do sistema como um todo \cite{Paigi2013}. No nível terciário, a coordenação abrangente entre unidades geradoras e consumidores é realizada. O objetivo principal é a minimização dos custos operacionais da microrrede, buscando o equilíbrio entre geração, consumo e demanda \cite{Paigi2013}. A integração desses níveis hierárquicos demonstra um potencial significativo para aprimorar a confiabilidade, eficiência e resiliência das microrredes. Essa abordagem robusta contribui para a viabilidade e o sucesso da integração de microrredes no sistema elétrico moderno.

% Tópico: Apresentação dos Desafios das Microrredes
Embora representem um avanço significativo na descentralização da geração de energia, as microrredes também apresentam desafios específicos no que diz respeito ao controle. A ausência de inércia rotativa, crucial nos sistemas elétricos tradicionais, é um dos principais obstáculos. Nas microrredes, a predominância de elementos interconectados eletronicamente contribui para essa falta de inércia, impactando a estabilidade do sistema \cite{Paigi2013}. Outro desafio significativo é a dependência da amplitude de tensão nas linhas de energia das redes de baixa ou média tensão. Essa característica exige estratégias de controle adaptáveis para garantir a distribuição eficiente da potência ativa \cite{Paigi2013}. Por fim, a intermitência das fontes renováveis, como a solar e a eólica, impõe um terceiro desafio. A natureza variável da geração de energia exige soluções inovadoras para o controle dinâmico da microrrede, assegurando a estabilidade e confiabilidade do sistema \cite{Paigi2013}. Superar esses desafios é fundamental para a operação eficiente das microrredes. A busca por soluções inovadoras no controle de energia elétrica torna-se crucial para o desenvolvimento e a implementação bem-sucedida dessa tecnologia.

% Tópico: Apresentação dos Desafios das Microrredes: REDs
A integração de \acrfullpl{red}, como painéis solares e turbinas eólicas, em microrredes isoladas, que operam sem conexão à rede elétrica principal, impõe desafios adicionais ao controle do sistema. A necessidade de garantir o compartilhamento eficiente de potência e a estabilidade da rede se torna ainda mais crítica nesse contexto. Esquemas de controle hierárquico com foco na provisão de reserva surgem como soluções promissoras para microrredes isoladas. A reserva, definida como a capacidade de fornecer potência adicional para compensar eventos inesperados, é crucial para a robustez do sistema.  \cite{Paigi2013}. Ela pode ser dividida em reserva pré-primária e reserva primária, ambas explorando as características dos elementos da microrrede. A reserva pré-primária, utilizando a inércia rotacional dos geradores, contribui para a estabilidade inicial da rede. Já a reserva primária envolve fontes controláveis que se ajustam em tempo real para fornecer potência adicional conforme necessário. A provisão adequada de reserva garante a continuidade do fornecimento de energia, mesmo diante de falhas ou variações na demanda, promovendo a confiabilidade e a resiliência das microrredes isoladas. \cite{Paigi2013}.

% Tópico: Redes Elétricas Inteligentes e Integração de Recursos de Comunicação
As redes elétricas inteligentes, conhecidas como \textit{smart grids}, são redes que cada vez mais são constituídas por \acrshortpl{red}, pois contribuem para a otimização da eficiência e sustentabilidade energética. No entanto, isto também apresenta desafios de controle e operação, como a restauração de frequência e tensão, o compartilhamento de potência ativa e reativa, o balanceamento do estado de carga das baterias, a operação econômica ideal e a comutação suave \cite{Zhou2020}. As microrredes surgem como soluções promissoras para superar esses desafios, possibilitando a interconexão e formação de microrredes em rede (\acrshort{nmg}, do inglês \textit{Networked Microgrids}). As \acrshort{nmg} ampliam ainda mais os benefícios das microrredes, mas exigem alta confiabilidade nas redes de comunicação entre suas unidades. Problemas de comunicação podem comprometer o desempenho do controle do sistema, resultando em perda de eficiência e instabilidade operacional. \cite{Zhou2020}.

% Tópico: Conversores CC - Introdução
Para garantir o funcionamento eficiente e estável das microrredes, a importância dos conversores \acrshort{cc}-\acrshort{cc} assume um papel crucial. Esses dispositivos desempenham a função de adaptar as características das diferentes fontes de energia e cargas conectadas à microrrede, e podem garantir a compatibilidade entre os diversos elementos da microrrede possibilitando sua integração e operação eficiente. Além de proporcionar flexibilidade operacional, os conversores \acrshort{cc}-\acrshort{cc} também contribuem para aumentar a autonomia e confiabilidade das microrredes. Isso se torna especialmente relevante em situações de operação isolada da rede elétrica principal, onde a microrrede precisa operar de forma autônoma e garantir o fornecimento de energia de forma segura e confiável. \cite{bessa2022}.

% Tópico: Conversores CC - Definição
Um sistema conversor \acrshort{cc}-\acrshort{cc} consiste em um circuito capaz de transferir energia elétrica de uma fonte de entrada para uma fonte de saída. Este circuito é composto por semicondutores de potência, que atuam como interruptores, e por elementos passivos, como indutores e capacitores, que controlam o fluxo de energia. A variável de controle fundamental do sistema é a razão cíclica, também conhecida como ciclo de trabalho. O objetivo principal do conversor \acrshort{cc}-\acrshort{cc} é regular a transferência de energia da fonte de entrada para a fonte de saída, adaptando-a às suas necessidades específicas. As fontes de entrada e saída podem variar consideravelmente, dependendo da aplicação do conversor. A carga conectada ao conversor também apresenta grande diversidade. Em alguns casos, pode ser composta por resistores. Em outros, pode ser um motor de corrente contínua, um banco de baterias, um dispositivo de soldagem elétrica a arco ou outro tipo de conversor estático. \cite{martins2008}.


% Tópico: Conversores CC - Classificação
A diversidade de topologias de conversores \acrshort{cc}-\acrshort{cc} se traduz em diversas opções para diferentes aplicações. Entre os tipos mais populares e amplamente utilizados, destacam-se: Buck, Boost, Buck-Boost, Cuk, Sepic e Zeta. Cada um possui características e funcionalidades específicas, atendendo a necessidades distintas de conversão de energia. O conversor Buck se destaca por sua simplicidade e eficiência na redução de tensão. A tensão na carga é sempre menor que a tensão da fonte de entrada. Por outro lado, o conversor Boost oferece a funcionalidade inversa, amplificando a tensão de entrada e fornecendo uma tensão mais alta na carga. Para aplicações que exigem flexibilidade na conversão de tensão, os conversores Buck-Boost, Cuk, Sepic e Zeta são soluções versáteis. Eles podem operar tanto como redutores quanto ampliadores de tensão, adaptando-se às necessidades específicas do sistema. O conversor Buck é o único que apresenta uma relação linear entre a tensão de entrada e a tensão de saída, o que facilita o controle e a previsibilidade do sistema. \cite{martins2008}.

% Tópico: Controle Acionado por Evento - Introdução
Nos cenários abordados até o momento, especialmente em ambientes onde recursos de comunicação e energia são críticos, uma gestão cuidadosa se faz necessária para mitigar potenciais problemas que possam comprometer o funcionamento do sistema. Nesse contexto, o \acrfull{etc} - emerge como uma solução promissora para o gerenciamento das \acrshortpl{mr} \acrshort{cc}. Ao contrário das metodologias tradicionais que empregam intervalos de amostragem e comunicação fixos, o \acrshort{etc} adapta esses processos em resposta a eventos específicos, como mudanças na demanda de energia ou no estado da rede. Essa adaptabilidade possibilita uma utilização mais eficiente dos recursos disponíveis, reduzindo desperdícios ao realizar amostragens e comunicações somente quando necessário. Assim, o \acrshort{etc} pode contribuir significativamente para a redução do consumo de energia, da carga de comunicação e do processamento computacional dos sistemas. \cite{coutinho2021}.

% Tópico: Controle Acionado por Evento - Classificações - Dinâmico e Estático
As estratégias de \acrshort{etc} podem ser classificadas em dois tipos principais: estáticas e dinâmicas (ou adaptativas). Essa classificação se baseia na função de ativação presente no \acrshort{etm}, que é responsável por determinar quando os eventos de controle do sistema serão acionados. No \acrshort{etc} estático, a função de ativação é fixa e depende apenas do estado atual do sistema e do último estado transmitido. Isso significa que os eventos de controle são predefinidos e não se adaptam às mudanças nas condições do sistema. \cite{coutinho2021}. Por outro lado, no \acrshort{etc} dinâmico ou adaptativo, a função de ativação incorpora uma variável dinâmica interna adicional. Essa variável pode ser um relógio interno ou outra variável dinâmica que funciona como um relógio, e sua taxa de crescimento está relacionada ao estado do sistema. \cite{Girard2015}.

% Tópico: Controle Acionado por Evento - Classificações - Emulação e Co-design
Para o projeto de sistemas \acrshort{etc}, há duas abordagens principais distintas: emulação e co-design. A emulação divide o projeto em duas etapas: primeiro, o controlador é projetado para garantir a estabilidade e o desempenho do sistema em malha fechada, ignorando o \acrshort{etm} e a rede de comunicação. Em seguida, o \acrshort{etm} é projetado considerando o controlador pré-existente e os efeitos da rede de comunicação. Essa abordagem oferece flexibilidade no projeto do \acrshort{etm}, mas pode limitar o desempenho em malha fechada e exigir mais transmissões do que o necessário. Por outro lado, o co-design envolve o projeto simultâneo do \acrshort{etm} e da lei de controle. Essa abordagem supera as limitações da emulação e otimiza o desempenho global do sistema. No entanto, o co-design é um desafio devido a problemas de otimização não convexas ou multiobjetivo, e a análise geralmente é limitada a classes específicas de controladores e \acrshortpl{etm}. \cite{coutinho2021}.


\section{Objetivos}

\subsection{Objetivos Gerais}

Este estudo, com base nos contextos delineados acerca das \acrshortpl{mr} e dos conversores \acrshort{cc}-\acrshort{cc}, propõe avaliar diversas estratégias de \acrshort{etc} para o controle de conversores Buck e Boost. Esta proposta abrange a implementação de modelos abrangentes que levam em conta a regulação de tensão e aprimoram a qualidade de energia, com o objetivo de fomentar a estabilidade local, facilitar o compartilhamento eficiente de potência e otimizar a eficiência econômica, no contexto das \acrshortpl{mr}. Tais contribuições exploram novas perspectivas no campo do controle e destacam a importância de estratégias formais para garantir a segurança, eficiência e adaptabilidade desses sistemas, os quais estão em constante evolução.

\subsection{Objetivos Específicos}
Para alcançar o objetivo geral, serão delineados os seguintes objetivos específicos que direcionarão as atividades do projeto:

\begin{enumerate}
  \item[(1)] Desenvolver modelos específicos dos conversores Buck e Boost
  \item[(2)] Implementar um simulador do conversor a partir do modelos desenvolvidos
  \item[(3)] Projetar diferentes ETMs e controladores para os conversores
  \item[(4)] Avaliar as diferentes estratégias de ETC para o controle dos conversores
\end{enumerate}

\section{Organização do Trabalho}

O presente trabalho encontra-se estruturado em quatro capítulos dos quais, o primeiro é composto por esta introdução ao trabalho.

O Capítulo \ref{cap2}, "Revisão Bibliográfica", abrange a fundamentação teórica, destacando o Sistema de Controle ETC, Microrredes CC e LMIs. A segunda parte revisita Trabalhos Relacionados, focando em Controle Acionado por Eventos e Controle em Rede, estabelecendo bases para o projeto.

O Capítulo \ref{cap3}, "Modelos e Protótipo", destaca a criação de modelos específicos para microrredes CC e aborda o desenvolvimento do protótipo, essencial para a implementação e teste das estratégias de Controle Acionado por Eventos propostas.

O Capítulo \ref{cap4}, "Avaliação de ETC", concentra-se na análise e avaliação das estratégias de Controle Acionado por Eventos (ETC) implementadas. Este capítulo examina o desempenho das estratégias em microrredes de corrente contínua, considerando métricas como eficiência energética, confiabilidade do sistema e adaptabilidade às condições operacionais.

O Capítulo \ref{cap5}, "Conclusão", apresenta as principais descobertas do projeto, destacando conclusões derivadas da avaliação das estratégias de Controle Acionado por Eventos em microrredes de corrente contínua.

% Fim Capítulo