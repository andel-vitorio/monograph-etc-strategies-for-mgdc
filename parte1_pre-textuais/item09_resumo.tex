\chapter*{Resumo}
\thispagestyle{empty}

Com a crescente demanda por fontes de energia sustentáveis e eficientes, as \acrfullpl{mr} elétricas de \acrfull{cc} assumem um papel fundamental na descentralização da geração de energia. Caracterizadas pela geração distribuída e armazenamento local, elas representam uma referência inovadora na gestão energética. No entanto, a implementação de \acrshortpl{mr} \acrshort{cc} traz consigo desafios, especialmente no controle e integração dos recursos energéticos. Além disso, as redes elétricas têm evoluído, tornando-se cada vez mais inteligentes, e têm utilizado tecnologias avançadas para otimizar a eficiência e a sustentabilidade energética, onde os recursos de comunicação desempenham um papel crucial nesse processo, facilitando a coordenação e o controle dos elementos da rede. No entanto, a integração desses recursos também pode apresentar desafios, que precisam ser abordados para garantir o funcionamento eficiente das \acrshortpl{mr} de \acrshort{cc}. Para garantir o correto funcionamento das cargas e minimizar as perdas em uma \acrshort{mr} de \acrshort{cc}, é comum fazer uso de conversores \acrshort{cc}-\acrshort{cc}. Eles desempenham um papel fundamental na adaptação de diferentes fontes de energia e cargas, contribuindo para a autonomia e confiabilidade dos sistemas elétricos. Diante desses desafios, foi proposta uma condição suficiente para o projeto de estratégias de controle baseadas em eventos, utilizando a abordagem por \textit{co-design} para os conversores \acrshort{cc}-\acrshort{cc} \textit{Buck} e \textit{Boost}. A aplicação dessa metodologia visa reduzir o número de eventos acionados, garantindo a estabilidade e o desempenho dos conversores.

\vspace{50pt}

\paragraph{Palavras-chave: Microrredes, Conversores \acrshort{cc}-\acrshort{cc}, Controle Baseado em Eventos, \textit{Co-design}}.
