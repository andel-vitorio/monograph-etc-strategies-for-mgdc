\chapter*{Agradecimentos}
\thispagestyle{empty}

Primeiramente, agradeço a Deus por conceder-me saúde e força para enfrentar os desafios desta jornada, proporcionando-me discernimento para tomar decisões importantes e orientando-me em cada etapa. Graças à Sua graça, consegui superar minhas batalhas, mantendo fé e determinação mesmo nos momentos mais difíceis. Em segundo lugar, agradeço aos meus pais, Gleiciane e Marivaldo, a base sólida sobre a qual construí minha vida. Desde o início, eles me proporcionaram tudo o que eu precisava para crescer e me desenvolver, tanto material quanto emocionalmente. Seu amor e apoio incondicionais permitiram-me superar desafios, perseguir meus sonhos e alcançar meus objetivos.

Aos meus queridos amigos, especialmente a André Girão, Maria Regina, Bruno Solimões, Davilos Maclaus, Karen Letícia, João Victor e Caroline Soares, dedico a mais profunda gratidão por todo o apoio, companheirismo e amizade que me deram durante esta jornada. Vocês foram pilares fundamentais, me incentivando nos momentos de dúvida, celebrando minhas conquistas e me oferecendo um ombro amigo quando precisei. Graças à sua presença, meus desafios se tornaram mais leves e minhas alegrias se multiplicaram. A cada passo que eu dava, sentia a força da nossa amizade me impulsionando a seguir em frente. Sua convicção em minhas capacidades me deu a confiança necessária para superar meus limites e alcançar meus objetivos. Hoje, ao concluir este trabalho, sinto um orgulho imenso de ter compartilhado essa trajetória com vocês.

Ao professor Carlos Cruz, expresso minha enorme gratidão por seu apoio e dedicação durante os primeiros passos da minha jornada acadêmica. Seus ensinamentos e incentivos foram fundamentais que me conduziram até este ponto da minha trajetória. Ao professor-orientador Iury Bessa, agradeço pela paciência e parceria durante a orientação do presente trabalho. Sua expertise e disponibilidade me guiaram pelos desafios da pesquisa e me permitiram alcançar grandes resultados. Agradeço a ambos por acreditarem em meu potencial e por contribuírem de forma tão significativa para minha formação acadêmica.